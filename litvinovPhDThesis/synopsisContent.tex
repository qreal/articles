\subsection*{\Large Общая характеристика работы}
\fontsize{14pt}{15pt}\selectfont
\textbf{Актуальность темы.}
Разработка сложных программных систем, несмотря на несколько десятилетий 
развития программной инженерии, до сих пор остаётся непростой задачей. 
Связано это отчасти с тем, что программное обеспечение невидимо и нематериально, 
его трудно себе представить. Бороться с этой проблемой помогает визуальное 
моделирование --- подход, при использовании которого программа представляется в 
виде набора графических моделей, каждая из которых описывает её с разных точек 
зрения. Если при проектировании объектов реального мира используются чертежи, то при 
разработке сложных программных систем могут использоваться модели, описывающие разбиение системы на компоненты, 
протоколы взаимодействия объектов системы, и т.д. В отличие от чертежей, которые 
соотносятся с тем, как будет выглядеть в реальном мире проектируемый объект 
после того, как будет создан, вид визуальных моделей --- лишь предмет 
договорённости между разработчиками. Каждый человек может представлять себе 
программу по-разному, что создаёт дополнительные трудности при применении 
визуального подхода. Тем не менее, благодаря наличию стандартных широко 
распространённых графических языков, визуальное моделирование повышает 
продуктивность труда и качество результирующего продукта при разработке. 
Существует довольно большое количество исследований, подтверждающих это экспериментально.

Сейчас визуальные модели используются в основном при анализе и проектировании, 
а также как средство документирования и передачи информации между 
разработчиками. Однако же программы целиком или их фрагменты возможно 
автоматически генерировать по набору визуальных моделей, что позволяет 
непосредственно использовать результаты анализа и проектирования и в 
значительной степени автоматизировать труд программистов. 

При этом использование визуальных языков общего назначения, таких как UML, делает задачу 
разработки программного обеспечения только с помощью графических языков сложной
в силу наличия семантического разрыва между кодом и моделями. Такие языки работают в 
тех же терминах, в которых пишется исходный код на традиционных текстовых языках 
(классы, объекты, компоненты и т.д.), поэтому, чтобы полностью специфицировать 
поведение системы и сделать возможной автоматическую генерацию, модель должна 
содержать в себе столько же информации, что и исходный код программы, но это 
противоречит самому понятию модели как некоего упрощения моделируемого объекта. 
На самом деле, визуальная модель в этом случае даже менее удобна, чем код 
программы --- визуальные символы занимают на экране больше места, чем текст. 
Если же визуальная модель будет изображать только важные аспекты 
функционирования системы, опуская излишние подробности, то её можно будет 
сохранить обозримой и полезной для человека, но это сделает её бесполезной для 
исполнителя (например, для интерпретатора или генератора исходного кода). 
Именно так, в основном, используется UML сейчас --- как средство для анализа и 
дизайна системы, а сама система специфицируется ручным кодированием на текстовых 
языках. Большинство инструментов для рисования UML-диаграмм позволяют 
сгенерировать заглушки, куда предполагается дописывать код вручную, но 
существенного выигрыша для разработчиков это не даёт.

Существует принципиально другой подход к использованию визуального 
моделирования, называемый предметно-ориентированным моделированием. Он основан на том наблюдении, что иногда создать новый 
язык для какой-то узкой предметной области или даже для конкретной задачи и решить задачу 
на нём оказывается быстрее и эффективнее, чем решать эту задачу на языке общего назначения. 
В таком случае наличие у средств поддержки создаваемого языка знаний о предметной области 
позволяет добиться полной автоматической генерации программ про визуальным моделям.

Существует много широко известных текстовых предметно-ориентированных языков, 
например, язык для работы с данными SQL, язык для работы с текстами awk,
средства описания контекстно-свободных грамматик для генераторов синтаксических 
анализаторов. В каждом из этих примеров программа на предметно-ориентированном 
языке работает в терминах той предметной области, для которой этот язык создан,
что даёт возможность не задумываясь о деталях реализации решать требуемую 
задачу. В предметно-ориентированных языках знания о предметной области "`спрятаны"' 
в инструментальные средства, что позволяет существенно расширить круг пользователей языка, 
вплоть до того, что на нём смогут программировать люди, далёкие от программирования. Ряд исследований
показывает, что продуктивность труда программистов при использовании 
предметно-ориентированных языков вырастает в 3-10 раз по сравнению с 
использованием языков общего назначения, поэтому такой подход представляется 
весьма перспективным.

Разумеется, создавать новый предметно-ориентированный визуальный язык и 
инструментальные средства его поддержки "`с нуля"' для каждой узкой предметной 
области или конкретной задачи было бы неоправданно трудозатратно. Поэтому 
существуют специальные средства для автоматизации этой задачи, называемые 
"`DSM-платформа"', или "`MetaCASE-средство"'. Такие средства позволяют задать 
синтаксис визуального языка, используя какой-либо формализм (как правило, 
это метамодели), и автоматически сгенерировать редактор этого языка и другие средства инструментальной 
поддержки (мы будем называть результат генерации термином "`DSM-решение"'). 
Это позволяет реализовывать технологии программирования, использующие новые предметно-ориентированные языки, за время 
порядка дней, что делает предметно-ориентированное моделирование оправданным 
даже для небольших проектов. Существуют зрелые исследовательские и промышленные 
DSM-платформы, такие как Eclipse Modeling Project, MetaEdit+ и другие; однако же, 
несмотря на значительные преимущества предметно-ориентированного 
моделирования, применяется оно довольно редко. Связано это, в частности, с 
недостатками существующих платформ и отсутствием развитой методологической 
базы для их применения. Во многих случаях для создания предметно-ориентированного 
решения требуется привлекать экспертов в создании языков, которыми зачастую 
оказываются авторы выбранной для реализации этого решения DSM-платформы, поэтому 
позволить себе это могут лишь крупные компании. Такая ситуация указывает на 
необходимость продолжения исследований в этой области, что и стало предметом 
данной работы.

\textbf{Целью} данной работы является исследование предметно-ориентированных 
визуальных языков, их свойств, процесса их разработки, создание методологии 
разработки предметно-ориентированных решений, достаточно простой в применении, 
чтобы свой предметно-ориентированный язык и инструментальные средства для него 
мог создавать даже человек, не имеющий опыта и специальной подготовки в создании 
визуальных языков. Также требуется создать технологию, эту методологию 
реализующую, которая позволяла бы неспециалистам создавать свои 
предметно-ориентированные решения в короткие сроки. 

Для достижения поставленной цели необходимо было решить следующие \textbf{задачи}:
\begin{enumerate}
	\item провести обзор подходов к реализации визуальных языков и средств, их реализующих;
	\item разработать в рамках DSM-платформы QReal простую в использовании технологию создания предметно-ориентированных языков;
	\item реализовать разработанную технологию в виде промышленного продукта;
	\item провести апробацию технологии путём создания нескольких DSM-решений с её помощью.
\end{enumerate}

\textbf{Объектом исследования} являются визуальные языки, методы их создания и технологии
для разработки инструментальных средств визуальных языков.

В качестве \textbf{методов исследования} используются: теория формальных языков, теория графов, 
методы объектно-ориентированного программирования.

\textbf{Научная новизна} данной работы заключается в следующем:
\begin{enumerate}
	\item Разработана новая методика и набор инструментальных средств для создания предметно-ориентированных 
		языков с помощью графического языка метамоделирования и сопутствующих визуальных языков.
		Методика предполагает применение предметно-ориентированного подхода "`самого к себе"', то есть
		предметно-ориентированные языки используются для описания всей функциональности разрабатываемых
		инструментальных средств для нового языка: редактора диаграмм, генераторов кода на текстовых языках по 
		диаграммам, интерпретаторов, средства проверки ограничений на диаграммы, средства поддержки рефакторингов.
	\item Предложен новый способ создания предметно-ориентированного языка: "`метамоделирование на лету"'. 
		Способ предполагает изменение и дополнение визуального языка прямо в процессе создания диаграммы на нём,
		без использования отдельного метаредактора. В процессе разработки языка при таком подходе
		не требуется оперировать с понятиями "`метамодель"' и "`метаредактор"', что снижает 
		требования к квалификации пользователей.
	\item С использованием предложенных методик разработаны новые предметно-ориентированные языки и
		средства инструментальной поддержки для них: язык программирования роботов и среда QReal:Robots
		(также известная как TRIKStudio), средство программирования приложений для мобильных телефонов 
		QReal:Ubiq, средство разработки аппаратных систем QReal:HaSCoL.
\end{enumerate}

\textbf{Практическая ценность} данной работы определяется использованием полученных 
результатов при разработке DSM-платформы QReal, а также ряде DSM-решений, созданных с её помощью, 
самым зрелым из которых стала среда программирования роботов QReal:Robots (TRIKStudio), 
предназначенная для обучения школьников основам информатики и кибернетики с использованием робототехнических 
конструкторов ТРИК, Lego Mindstorms NXT, Lego Mindstorms EV3.

Среда QReal разрабатывается в рамках деятельности научно-исследовательской 
группы по изучению визуального моделирования под руководством проф. 
А.Н.~Терехова с 2007 года и базируется на более чем двадцатилетнем опыте 
коллектива кафедры системного программирования Санкт-Петербургского 
государственного университета в разработке графических языков. 
Проект имеет открытый исходный код\footnote{Страница проекта и репозиторий с исходным кодом на GitHub, URL: https://github.com/qreal/qreal}, 
разрабатывается на языке C++ с использованием библиотеки Qt силами студентов и преподавателей кафедры, автор 
данной диссертации --- один из руководителей проекта. QReal создаётся как 
средство визуального моделирования, поддерживающее ряд широкоизвестных 
визуальных языков (UML 2.0, BPMN, блок-схемы), и одновременно как DSM-платформа, 
позволяющая быстро и без специальных знаний создавать свои собственные 
визуальные языки и DSM-решения на их основе. На данный момент среда существует 
в виде работающего прототипа. Проект поддержан грантом Санкт-Петербургского 
государственного университета 6.39.1054.2012. DSM-платформа QReal использовалась
 для реализации ряда предметно-ориентированных решений, использовавшихся в 
проектах компании "`ЛАНИТ-Терком"', связанных с разработкой информационных систем 
и систем компьютерного зрения.

Среда программирования роботов QReal:Robots (или TRIKStudio) --- на данный момент наиболее зрелая 
предметно-ориентированная технология, созданная с помощью среды QReal. 
Первый прототип среды программирования был разработан автором данной диссертации
с использованием системы QReal примерно за неделю и включал в себя визуальный 
язык из примерно 20 сущностей, редактор к нему и интерпретатор, позволяющий 
исполнить программу на компьютере, посылая команды роботу Lego Mindstorms NXT по интерфейсу 
Bluetooth. На данный момент система переименована в TRIKStudio и обладает возможностями управления роботами 
ТРИК, Lego Mindstorms NXT, Lego Mindstorms EV3 по WiFi, Bluetooth, USB, генерации программы с 
последующей загрузкой её на робот для автономного исполнения и исполнения 
программы на двухмерной модели робота на экране компьютера.

\pagebreak

\textbf{Апробация работы} заключается в следующем.
\begin{itemize}
	\item Некоторые результаты данной работы были доложены на второй 
		научно-технической конференции молодых специалистов "`Старт в будущее"' 
		(Санкт-Петербург, 2011). Доклад был отмечен наградой.
	\item Результаты, связанные с применением разработанной технологии при 
		создании среды QReal:Robots, были доложены на VII Международной 
		научно-практической конференции "`Современные информационные технологии 
		и ИТ-образование"' (Москва, 2012).
	\item Результаты, связанные с применением разработанной технологии для 
		разработки предметно-ориентированного языка для платформы Ubiq, были доложены 
		на международной конференции "`10th Conference of Open Innovations 
		Association FRUCT"' (Tampere, 2011).
	\item Результаты диссертации использовались при проектировании и реализации 
		DSM-платформы QReal и ряда DSM-решений, созданных с её помощью, включая 
		среду программирования роботов QReal:Robots. Среда QReal использовалась для 
		создания нескольких предметно-ориентированных решений в ЗАО “ЛАНИТ-Терком”. 
		Среда QReal:Robots демонстрировалась на Открытых состязаниях 
		Санкт-Петербурга по робототехнике в 2012 году и на робототехническом 
		фестивале "`Робофест 2012"' в Москве. На данный момент эта среда переименована в 
		TRIKStudio и используется как основное средство программирования 
		кибертенического конструктора ТРИК, используется в нескольких робототехнических 
		кружках в России и на мастер-классах по робототехнике, проводимых компанией "`Кибернетические технологии"'.
	\item По теме диссертации опубликовано пять научных работ (три из них ---
		в сборнике из списка ВАК, две другие --- в сборнике, входящем в РИНЦ) и 
		десять тезисов докладов на конференциях (под авторством или в соавторстве с 
		автором диссертации).
\end{itemize}

\textbf{Структура и объём работы.} Диссертация состоит из введения, пяти глав, заключения, 
списка сокращений и условных обозначений, списка литературы (81 наименование). Объем основной 
части работы --- 149 страниц с 34 рисунками и 3 таблицами.

\subsection*{\Large Содержание работы}
Во \textbf{введении} обосновывается актуальность исследований, проводимых в рамках 
данной диссертационной работы, формулируется цель работы, научная новизна и практическая 
значимость, приводятся сведения об апробации работы.

В \textbf{Главе 1} приводятся основные понятия, используемые в 
предметно-ориентированном визуальном моделировании. Обсуждается структура 
визуального языка: синтаксис (абстрактный, конкретный и синтаксис сериализации), 
уровни абстракции (предметная область, модель, метамодель, метаметамодель), вводится 
классификация формальных языков по важным для дальнейшего изложения свойствам.

По форме представления информации языки делятся на следующие группы: \textit{текстовые} языки 
--- используют в основном текст для представления информации, \textit{графические} или 
\textit{визуальные} языки, использующие визуальные символы, и \textit{текстографические} языки,
активно использующие и текст, и графику для представления программы. Графические языки, в свою очередь, 
делятся на графовые (в которых диаграмма представляет собой помеченный мультиграф, либо сводится к нему)
и неграфовые (не сводящиеся к помеченному мультиграфу).

По используемой модели вычислений языки делятся на \textit{статические} (задающие структуру разрабатываемой системы) \
и \textit{динамические} (описывающие поведение), которые, в свою очередь, делятся на языки, 
модель вычислений которых основана на сетях Петри (с простыми токенами --- ориентированные на поток управления,
или с токенами, содержащими в себе существенную информацию --- ориентированные на поток данных),
и языки, в основе которых лежат диаграммы конечных автоматов.

\textbf{Глава 2} содержит обзор существующих подходов к созданию DSM-решений. 
Рассматриваются существующие методологии разработки предметно-ориентированных языков,
обсуждаются возможности, достоинства и недостатки существующих DSM-платформ, 
включая зрелые системы и академические разработки. Результаты анализа существующих 
DSM-платформ сведены в таблицу \ref{tab:existingPlatformsMain}.

\begin{table}[!ht]
\begin{small}
	\begin{longtabu} {| X[1 l p] | X[1 l p] | X[1 l p] | X[1 l p] |X[1 l p] |}
		\caption{Основные возможности существующих DSM-платформ} \\
		\tabucline-
		 Название                    & Метаязык                        & Метаредактор                                            & Конкретный синтаксис                                     & Ограничения                                    \\
		\tabucline-
		\everyrow{\tabucline-}
		MetaEdit+                    & GOPPRR                          & Визуальный или диалоговые окна                          & Визуально                                                & Только средствами метаязыка                    \\
		Eclipse Modeling Project     & Ecore (аналог MOF)              & Визуальный, текстовый, импорт метамодели                & Визуально или вручную поверх библиотек                   & Текстовые (OCL)                                \\
		Generic Modeling Environment & Свой, довольно развитый         & Визуальный                                              & Настройкой существующих фигур                            & Текстовые (OCL)                                \\
		PSL/PSA                      & Сущность-связь                  & Текстовый                                               & Нет                                                      & Нет                                            \\
		AToM3                        & Сущность-связь                  & Визуальный                                              & Визуально                                                & Вручную, на Python                             \\
		Microsoft Modeling SDK       & Свой (диаграммы классов)        & Визуальный                                              & Настройкой существующих фигур или кодированием на C\#    & Вручную поверх библиотеки                      \\
		Pounamu                      & Сущность-связь                  & Визуальный                                              & Визуально                                                & Визуальным языком общего назначения или Java   \\
		DOME                         & Свой (DSTL), довольно развитый  & Визуальный                                              & Настройкой существующих фигур или кодированием на Alter  & Вручную, на Alter                              \\
		MetaLanguage                 & Сущность-связь                  & Визуальный                                              & Неизвестно                                               & Да                                             
		\label{tab:existingPlatformsMain}
	\end{longtabu}
\end{small}
\end{table}

Делаются выводы о том, что все существующие среды не автоматизируют весь жизненный цикл
создания предметно-ориентированного решения, особенно его начальные этапы --- если автор DSM-решения уже 
"`знает, что писать"' (то есть имеет в голове ясное представление о предметной области 
и даже метамодель создаваемого языка), то к его услугам множество существующих инструментов. 
Но если требуется вести разработку предметно-ориентированного решения "`с нуля"' (начиная 
с оценки осуществимости и анализа предметной области), очень многое придётся делать 
без какой-либо инструментальной поддержки и, как правило, без руководства к действию.
Таким образом, явно имеется пробел в существующих исследованиях, который данная работа
призвана помочь заполнить.

\textbf{Глава 3} содержит описание предлагаемого подхода к разработке 
DSM-решений: приводятся этапы жизненного цикла DSM-решения, обсуждается 
возможная степень автоматизации каждого этапа, формулируются требования на 
средства автоматизации, приводится описание предлагаемой технологии. Предлагается
два варианта методологии разработки --- "`Классическая"' методология и методология "`Метамоделирования
на лету"', которая является вариантом классической и служит для упрощения первых этапов
жизненного цикла DSM-решения. Схематически "`Классическая"' методология изображена на рисунке
\ref{classicMethodology}.

\begin{figure} [ht]
	\begin{center}
		\includegraphics[width=0.25\textwidth]{part3/classicMethodology.png}
		\caption{"`Классическая"' методология разработки визуального предметно-ориентированного языка.}
		\label{classicMethodology}
	\end{center}
\end{figure}

Фаза разработки и внедрения состоит из нескольких итераций, порядок действий для каждой 
из которых представлен на рисунке~\ref{classicMethodologyIteration}. 

\begin{figure} [ht]
	\begin{center}
		\includegraphics[width=0.9\textwidth]{part3/classicMethodologyIteration.png}
		\caption{Итерация проектирования и разработки языка.}
		\label{classicMethodologyIteration}
	\end{center}
\end{figure}

Такая методология называется классической, поскольку примерно такой схемы придерживается
большинство существующих DSM-платформ и большинство авторов, описывающих процесс создания
предметно-ориентированных языков и дающих рекомендации по этому процессу. Вклад данного 
исследования состоит в структуризации этапов жизненного цикла, описании действий на 
каждом этапе жизненного цикла и предложения технологии автоматизации каждого этапа.
Основная идея, предлагаемая здесь --- использование визуальных языков на каждом
этапе жизненного цикла, вплоть до описания генераторов.

С методологической точки зрения научную новизну имеет предлагаемый здесь подход 
"`Метамоделирования на лету"'. Ключевой принцип данной методологии состоит в том, 
что создание визуального языка проходит непосредственно в процессе рисования диаграммы, 
без использования метаредактора.

Основной этап предлагаемой методологии --- прототипирование языка --- начинается сразу 
после этапа анализа применимости. Разработчик языка и будущий пользователь работают 
за одним рабочим местом. При запуске DSM-платформы они видят канву для рисования и 
пустую палитру. Пользователь объясняет, что он примерно хотел бы нарисовать, разработчик 
языка добавляет на палитру новые элементы, определяет для них графическое представление, 
пользователь рисует. Пользователь может сказать, что такой-то элемент должен содержать 
такую-то дополнительную информацию, тогда разработчик добавляет элементу новое свойство, 
задаёт его тип и значение по умолчанию, и пользователь продолжает рисовать диаграмму, 
используя новое свойство. Через некоторое время пользователь может сам добавлять и 
редактировать типы элементов, и работа полностью передаётся ему, разработчик языка 
лишь следит за процессом и консультирует при необходимости пользователя. Работа заканчивается, 
когда модельное приложение полностью нарисовано, после чего текущая интерпретируемая 
метамодель сохраняется в виде, пригодном для дальнейшего редактирования в метаредакторе. 
После этой фазы идут итерации "`классической"' методологии по доработке созданного 
прототипа, дополнению его ограничениями, рефакторингами, интерпретатором и генератором, 
подготовки инсталляционного пакета, развертывания и сбора обратной связи. На этих этапах 
пользователи, как и в "`классической"' модели, непосредственно в разработке не участвуют, 
поскольку этапы гораздо более продолжительны во времени и прямо при пользователе выполнены 
быть не могут.

Модель жизненного цикла языка, использующая "`метамоделирование на лету"', представлена 
на рисунке~\ref{metamodelingOnFly}.

\begin{figure} [ht]
	\begin{center}
		\includegraphics[width=0.25\textwidth]{part3/metamodelingOnFly.png}
		\caption{Методология "`метамоделирования на лету"'.}
		\label{metamodelingOnFly}
	\end{center}
\end{figure}

В \textbf{Главе 4} анализируются результаты реализации инструментальных средств поддержки предлагаемой 
технологии в проекте QReal. Описываются возможности системы QReal, связанные с поддержкой 
техник метамоделирования, включая метамоделирование на лету, принятые архитектурные 
решения.

\textbf{Приложение A} содержит примеры применения результатов, описанных в данной 
диссертации, для разработки DSM-решений. Описывается среда QReal:Robots, то, 
какие преимущества были получены от использования DSM-платформы QReal при её 
разработке, то, чем QReal помочь не смог, и почему. Также приводится 
описание среды разработки сервисов для мобильных телефонов QReal:Ubiq и среды
разработки аппаратуры QReal:HaSCoL, описываются их визуальные языки, 
достоинства и недостатки принятых при их создании подходов.

В \textbf{Приложении B} описывается визуальный метаязык системы QReal.

Реализация предложенных в данной диссертации методик осуществлялась коллективом студентов 
и аспирантов кафедры системного программирования СПбГУ в рамках проекта QReal. Хочется 
особо отметить вклад студентов, работавших над данным проектом под руководством автора 
диссертации: Абрамова Ивана Александровича, Дерипаска Анны Олеговны, Гудошниковой Анны Андреевны, 
Жуковой Беллы Юрьевны, Заболотных Елены Петровны, Занько Софьи Владимировны, Иванова Всеволода Юрьевича, 
Кузенковой Анастасии Сергеевны, Кузнецовой Марьи Юрьевны, Курбанова Рауфа Эльшад оглы, 
Назаренко Владимира Владимировича, Нефёдова Ефима Андреевича, Никольского Кирилла Андреевича, 
Осечкиной Марии Сергеевны, Птахиной Алины Ивановны, Пышновой Александры Витальевны, 
Савина Никиты Сергеевича, Соковиковой Натальи Алексеевны, Такун Евгении Игоревны, 
Тихоновой Марии Валерьевны, всех студентов, работавших под руководством Брыксина Тимофея Александровича, 
а также вклад Дмитрия Мордвинова и Ирины Брюхановой. Без них данная диссертация вряд ли 
была бы возможна.

В \textbf{заключении} приведены основные результаты работы.

\subsection*{\Large Основные результаты работы}
\begin{enumerate}
	\item Проведён анализ различных подходов к реализации визуальных языков и различных 
		технологических средств, их реализующих.
	\item Разработана методика и набор инструментальных средств для  создания предметно-ориентированных 
		языков с помощью графического языка метамоделирования и сопутствующих визуальных языков.
	\item Предложен новый способ метамоделирования: "`метамоделирование на лету"'.
	\item Предложенные методики и технологии реализованы в виде промышленного продукта QReal.
	\item Проведена апробация при создании редактора, генератора, средств проверки ограничений 
		среды QReal:Robots и других предметно-ориентированных решений.
\end{enumerate}

\subsection*{\Large Публикации автора по теме диссертации}

\nocitevak{*}
\bibliographyvak{myVakWorks}

\renewcommand{\refname}{\Large В других изданиях}
\nocite{*}
\bibliography{myOtherWorks}
