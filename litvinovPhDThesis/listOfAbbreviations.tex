\chapter*{Список сокращений и условных обозначений}
\addcontentsline{toc}{chapter}{Список сокращений и условных обозначений}

\begin{acronym}
	\acro{UML}{Unified Modeling Language, URL: http://uml.org/ (дата обращения: 22.02.2014г.)}
	\acro{DSM}{Domain Specific Modeling}
	\acro{DSL}{Domain Specific Language}
	\acro{ER}{Entity-Relationship}
	\acro{dsmPlatform}[DSM-платформа]{Инструментарий для создания предметно-ориентированных средств разработки (DSM-решений)}
	\acro{metaCaseTool}[metaCASE-средство]{Здесь используется как синоним термина <<\ac{dsmPlatform}>>, 
		получил широкое распространение после выхода в свет статьи~\cite{alderson1991meta},
		но в данной диссертации этот термин почти не используется, в силу расплывчатости понятия <<CASE>>}.
	\acro{dsmSolution}[DSM-решение]{Предметно-ориентированный 
		язык и технологические средства для работы с ним (например, редактор, генераторы, верификаторы и т.д.)}
	\acro{BPMN}{Business Process Model and Notation, URL: http://www.bpmn.org/ (дата обращения: 22.02.2014г.)}
	\acro{Bluetooth}{Спецификация беспроводных сетей ближней связи, 
		URL: https://www.bluetooth.org/en-us/specification/adopted-specifications (дата обращения: 22.02.2014г.)}
	\acro{USB}{Universal Serial Bus, <<универсальная последовательная шина>>, интерфейс передачи данных для периферийных устройств
		URL: http://www.usb.org/ (дата обращения: 22.02.2014г.)}
	\acro{Software}[ПО]{Программное обеспечение}
	\acro{CASE}{Computer-Aided Software Engineering, термин, которым исторически обозначают среды визуального моделирования и программирования.}
	\acro{XML}{Extensible Markup Language. Стандарт 1.0, URL: http://www.w3.org/TR/REC-xml/ (дата обращения 21.08.2014г.).}
	\acro{FPGA}{Field-Programmable Gate Array}
	\acro{SQL}{Structured Query Language}
	\acro{DDL}{Data Definition Language}
	\acro{DBMS}[СУБД]{Система управления базами данных}
	\acro{API}{Application Programming Interface}
\end{acronym}